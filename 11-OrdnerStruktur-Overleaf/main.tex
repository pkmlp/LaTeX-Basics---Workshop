\documentclass[12pt, a4paper]{article}

\usepackage[utf8]{inputenc}   % für deutsche Umlaute (input)
\usepackage[ngerman]{babel}   % für neue deutsche Rechtschreibung 
\usepackage[T1]{fontenc}      % für deutsche Umlaute (output)

\usepackage{amsmath}          % für mathematische Formeln
\usepackage{amssymb}          % für mathematische Symbole

\usepackage{graphicx}         % für Bilder

\usepackage{csquotes}                           % für Literaturverzeichnis und Zitierungen
\usepackage[style=numeric]{biblatex}            % für Literatur-Verweise im Stil [1]
%\usepackage[style=alphabetic]{biblatex}         % für Literatur-Verweise im Stil [Kot11]
\addbibresource{./Verzeichnisse/Literatur.bib}  % Datei mit Literatur/Quellen


\setlength{\parindent}{0em}   % Einzug (Einrückung) der ersten Zeile im Abschnitt
\setlength{\parskip}{0.75em}  % Abstand zwischen den einzelnen Abschnitten

\title{\LaTeX-Intro}          % Titel der Arbeit, wird für \maketitle benötigt
\author{Peter Kessler}        % Autor der Arbeit, wird für \maketitle benötigt
\date{März 2019}              % Datum der Arbeit, wird für \maketitle benötigt


\begin{document}

\maketitle
\thispagestyle{empty}         % zur Unterdrückung der Seitenzahl auf der Titelseite

\pagebreak
\tableofcontents


\pagebreak 
\section{Herkunft}
\input{./Inhalte/Herkunft.tex}


\pagebreak
\section{WYSIWYG vs WYSIWYAF}

\subsection{WYSIWYG}
\input{./Inhalte/WYSIWYG.tex}

\subsection{WYSIWYAF}
\input{./Inhalte/WYSIWYAF.tex}


\pagebreak
\section{Arbeiten mit mathematischen Formeln}
\input{./Inhalte/Math.tex}

\subsection{Inline Formeln}
\input{./Inhalte/MathInline.tex}

\subsection{Abgesetzte Formeln}
\input{./Inhalte/MathAbgesetzt.tex}

\subsection{Ausgerichtete Formeln}
\input{./Inhalte/MathAusgerichtet.tex}


\pagebreak
\section{Zitieren und Literaturverzeichnis}

Quellenangaben und Zitate erfüllen in wissenschaftlichen Texten bzw. in wissenschaftlichen Arbeiten zwei Hauptziele\cite{NW2017}: 

\begin{itemize}

\item[a)] Nachvollziehbarkeit: Ein wissenschaftlicher Beitrag kann nur dann weiterverwendet werden, wenn sich die Argumentation für die Lesenden überprüfen lässt.

\item[b)] Unterscheidung zwischen eigenen und fremden Gedanken/Ideen: Bilder, Grafiken, Tabellen, Texte die aus anderen Arbeiten stammen, müssen als fremdes geistiges Eigentum ausgewiesen werden.

\end{itemize}

\textbf{Wichtige Regel:} Quellenangaben gehören zum Satz dazu und stehen daher VOR dem Punkt bzw. Satzzeichen.



\pagebreak
\addcontentsline{toc}{section}{Literatur-/Quellenverzeichnis}    % Eintrag in ToC erstellen
\printbibliography[title={Literatur-/Quellenverzeichnis}]        % Literaturverzeichnis erstellen

\end{document}
